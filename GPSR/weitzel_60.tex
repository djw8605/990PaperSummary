\documentclass[12pt]{article}


\begin{document}

\begin{center}
{\huge Summary: GPSR: Greedy Perimeter Stateless Routing for Wireless Networks } \\
Derek Weitzel
\end{center}

This paper describes a new routing protocol, GPSR, which routes packets based on topology information.

The authors say:
\begin{quote}
We further assume bidirectional radio reachability.
\end{quote}
From our other readings, this is traditionally not a good assumption.  But since the authors only simulated the network, probably didn't see any problems with this assumption.

The interface queue transversal seems very intuitive and logical, though I hadn't read about it before.  It's those findings that are found intuitive after that are usually the most interesting. 

Planarization of the graph seems like it may be a bad assumption.  Though true of most Wireless Sensor Networks, a specific example that they gave, rooftop network, it may not work in a large city with tall and short buildings next to each other.

They compared their routing vs DSR.  DSR has long been known to be bad on the Internet, is it widely used in WSN's?

\end{document}

