\documentclass[12pt]{article}


\begin{document}

\begin{center}
{\huge Summary: Overview of Sensor Networks } \\
Derek Weitzel
\end{center}


This paper was a very general overview of the field of Sensor Networks.  I have several comments on the paper.

In the first section, they say:
\begin{quote}
The individual devices in a wireless sensor network (WSN) are inherently resource constrained: They have limited processing speed, storage capacity, and communication bandwidth.
\end{quote}
I see this statement as short-sighted.  The history of computing is that the computers get smaller, and smaller.  At the same time, WSN's are getting more powerful as components from the computing world are miniaturized.  It may be true that WSN nodes, by definition, may be less powerful than desktop computers in the foreseeable future.  But It may also be true that desktop computers will become smaller and smaller until they are the size of a WSN node.

In the Sensor Network Applications section, they talk about what wireless sensor do.  I think one of the great benefits that can come out of the Sensor Network arena is complex mesh networking.  This has many applications all around us, but it isn't mentioned here, or in the paper, how advances in Sensor Networking can improve other fields.

One of the example applications was temp. and humidity on a large tree.  I was disappointed that they did not cover sensor network problems they had.  For example, I presume interference increased when the wind blew (moving leaves randomly).   Further, they don't talk about how the sensors communicated with the ground units?  Even though they talked about the old method, where there was a hard-wired connected base station on the ground.

In the Microradios section, the authors assert:
\begin{quote}
Thus, WSNs process data within the network wherever possible.
\end{quote}
They authors never define what they mean by 'process data within the network'.  I suspect it mean that processing is done by other, possibly nearby, WSNs?  Or maybe that the WSNs will process the data on their own, then send the processed data as output.



\end{document}

