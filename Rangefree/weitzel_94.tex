\documentclass[12pt]{article}


\begin{document}

\begin{center}
{\huge Summary: Achieving Range-free Localization Beyond Connectivity} \\
Derek Weitzel
\end{center}

This paper is a description how existing localization methods can be improved by incorporating a calculated factor related to the relative distance of the node.

I liked the beginning, that there where experiments that proved the assumptions used in the paper.  I am curious though, why these experiments that seem to prove something very fundamental, where not their own paper?  It seems that these experimental findings can be used more generally, and therefore should be their own paper.  Are these findings not useful otherwise?  Seems like 2 papers stapled onto each other.

This paper was very clever.  I especially like the bisector lines in 4.2.2.  It is both clever, and clear.

In the protocol, it replace a factory with the RSD.  The factor before was an binary integer, but RSD is a float.  Would this significantly increase to complexity to calculate the distance?

The paper mentions that the simulations where done 50 times to be statically significant.  I like that the paper says something about being statistically significant.  


\end{document}

