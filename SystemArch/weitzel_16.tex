\documentclass[12pt]{article}


\begin{document}

\begin{center}
{\huge Summary: System Architecture Directions for Networked Sensors } \\
Derek Weitzel
\end{center}

This paper is a survey paper of the current state of nodes, and describing a new operating system, TinyOS, to run on nodes.  The paper was very informative about the TinyOS operating system.  I found the introduction sections very generic, all of the descriptions could describe any node or small computer.

Just a quick note, with the 512 bytes of ram, the entire contents of RAM can be transferred out of the radio in .2s.  A very different ratio in than in a traditional computer.

There was a long description of the lack of a buffer in the radio, even describing how the radio wasn't latched, and noise can cause issues.  But, there was no description of a buffer in the serial port.  In systems I have used, the serial port has been a fire and forget device.  But on this device, is says it's asynchronous, but not how much of a buffer there is.

A note about the structure of the paper, the figures where usually 2 pages away from the first mention.  Made me go search for the picture, flipping back and forth between the description and the picture.

A method for more efficient temperature taking would be to trigger an interrupt on a threshold temperature change, or a timeout, whichever first.  I'm sure this type of technique has been used in visual sensors, infrared or otherwise.

They talk a lot about the data size is small.  But they don't say the code size is small, or mention how easy or hard it is to code in this OS.  This reminds me of ultrafast kernels written in assembly, but they where never maintainable because of complexity.  Also, they don't talk about where the code sits.  The description of the node describes the flash space as code, but an earlier paper said that the flash is very expensive to read.  So does this mean that the code is always read from the flash memory?  Where does the program information go?  Especially since given the size of the ram, it will likely be read more often than RAM.

\end{document}

