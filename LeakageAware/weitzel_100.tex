\documentclass[12pt]{article}


\begin{document}

\begin{center}
{\huge Summary: Leakage-Aware Energy Synchronization for Wireless Sensor Networks } \\
Derek Weitzel
\end{center}

This paper is an in-depth look at using ultra-capacitors in wireless sensor networks.  The paper has many EE concepts in it, such as capacitor capacity (F) and capacitor leakage.

There are many references to the Twinstar platform.  Is it a commercial platform?  Something you made?

The two power supplies for the smart circuit is clever, as it draws on what other systems have used.  For example, car batteries store energy to start the engine.  Heaters always have the pilot on in order to start the rest of the furnace.

I like that the metric for the system is to tell it how long it should last, and have it calculate what it needs to do to meet this demand.  Seems like real-time scheduling.  But, like in real-time scheduling, can the system just say 'no', that it can't meet that demand?

The calculations to predict lifetime are floating point calculations.  But there is no talk on what the compromise it for these calculations vs. some sort of integer calculation?  Can this be done only in integer terms?

The local energy synchronization makes it sound like the nodes will coordinate to all have the same energy levels.  But that is referred to as global energy synchronization, which I really like.  The local energy synchronization should be something like energy consumption suggestions, where the node asks the application to perform these actions to conserve energy.

How do you implement an oracle system?  Especially in the real world?

\end{document}

