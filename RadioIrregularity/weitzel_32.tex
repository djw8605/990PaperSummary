\documentclass[12pt]{article}


\begin{document}

\begin{center}
{\huge Summary: Impact of Radio Irregularity on Wireless Sensor Networks } \\
Derek Weitzel
\end{center}

This paper gives an overview of Radio Irregularity in low power sensor nodes.  It mostly talks about directional differences, such as sensors that have stronger signals at $90^{\circ}$ than at $270^{\circ}$.

The authors say radio irregularity is caused by 2 factors, devices and the propagation media.  But the authors seem to only focus on the devices argument, and leave the propagation media to only be 'air'.  This is particularly untrue of Nebraska use of WSN's, as there is large research about underground wireless sensors.  Also the authors mention reflection, refraction due to obstacles, but the baseline test they use is an empty parking lot.  Hard for me to think of a more obstacle free area.

The authors use the term signal path loss without defining it.  Does this mean loss on the 'path' between 2 nodes?

Authors do not attempt to answer the question as to why the South direction is stronger.  Are all the nodes facing the same direction?  Also, why is East so much worse than South?

The graphs of the battery usage effects are heavily zoomed in, while the effects of different motes are not as zoomed in.  I suspect that the graphs might look similar if they where similarly zoomed.  Also, the difference between 59.5 and 58 is 2.5\%, seems like a very insignificant difference, maybe not even statistically significant.

It appears in the MAC section that there is a tradeoff, between larger loss ration but better average single hop delays, or less loss ration and longer delays.  It seems to me that in low power environments, retransmitting is more expensive then just waiting for the packet to get somewhere.  The choice seems clear.

In the handshaking talk in MAC layer, they talk a lot about the hidden node problem.  To me it doesn't seem like a large deal, especially if the protocol has a 2 way ack.  Or even if it doesn't, the sender will just send again, to the same or another node anyways.  In low power, this might be significant though.

\end{document}

